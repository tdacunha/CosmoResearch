\documentclass[aps,nofootinbib,notitlepage,superscriptaddress,twocolumn,10pt,prd]{revtex4-1}

\usepackage{graphicx}
\usepackage{amsmath,amssymb}
\usepackage{hyperref}
\usepackage{braket}
\usepackage{subfigure}
\usepackage{float}
\usepackage[dvipsnames]{xcolor}
\usepackage{mathrsfs}
\usepackage{comment}
\usepackage{multirow}

\newcommand{\mr}[1]{\textcolor{ForestGreen}{#1}}

\begin{document}

\title{Measured cosmological parameters}

\begin{abstract}
%
This is the abstract
%
\end{abstract}

%
\maketitle
%

Some useful references \cite{Raveri:2018wln,Raveri:2019gdp,Raveri:2021wfz}.

The algebra absolute reference \url{https://www.math.uwaterloo.ca/~hwolkowi/matrixcookbook.pdf}.

Some notation.

We assume that both the posterior, likelihood and prior are Gaussian distributions.
We indicate parameters with $\theta$.
We indicate posterior quantities with the subscript $p$ so, for example, $\theta_p$ and $\mathcal{C}_p$ are the posterior means and covariance respectively.
We indicate prior quantities with the subscript $\Pi$ so, for example, $\theta_\Pi$ and $\mathcal{C}_\Pi$ are the prior means and covariance respectively.
We use no subscript for the likelihood so its means and covariance are just given by $\theta$ and $\mathcal{C}$.

Since everything is Gaussian we can write the posterior covariance as:
\begin{align}
\mathcal{C}_p^{-1} = \mathcal{C}^{-1} + \mathcal{C}_\Pi^{-1} 
\end{align}
and the posterior mean as:
\begin{align}
\theta_p = \mathcal{C}_p ( \mathcal{C}^{-1}\theta + \mathcal{C}_\Pi^{-1} \theta_\Pi)
\end{align}
\mr{Tara, is it clear why?}

In practice the Gaussian prior is expected to be exact for some parameters while approximate for parameters that have flat priors.
But it still captures two key aspects of a potentially informative flat prior, i.e. changing the center and scale of the posterior wrt to the likelihood.

We have some general requirements on the techniques that we should use to quantify measured parameters:
\begin{enumerate}
\item the parameter basis that we use should not matter, i.e. physics does not care about the way we write it down.
This means that the combination of parameters we identify in a parameter basis should change accordingly to the base change but should not change in nature.
\item we should take into account the prior in the analysis, the prior lives in our mind, we are interested in what the data is constraining.
This means we have a problem since we cannot really access the likelihood and we only have posteriors.
\end{enumerate}

%
\subsection{PCA}
%
In this section we show what happens for principal component analysis (PCA).

This amounts to computing the eigenvalues of the covariance matrix, i.e.:
\begin{align}
\mathcal{C}_p Q = Q \Lambda
\end{align}
where $Q$ is an orthogonal matrix $Q^{-1} = Q^T$, since $\mathcal{C}_p$ is real and symmetric and $\Lambda$ is a diagonal matrix with the eigenvalues.

Note the following properties:
\begin{enumerate}
\item $\mathcal{C}_p = Q \Lambda Q^T$
\item $\mathcal{C}_p^{-1} = Q \Lambda^{-1} Q^T$ where $\Lambda^{-1}$ is diagonal with $1/\lambda_i$ on the diagonal. This means that the PCA modes for the covariance and its inverse are the same.
\end{enumerate}

This has some problems. The first is that it is not invariant under an affine transformation.
Consider a linear reparametrization:
\begin{align}
\theta^\prime = A \theta
\end{align}
where $A$ is an invertible base change matrix. In general $A$ mixes and rescales the values of the parameters.
Under such a transformation the posterior covariance changes as:
\begin{align}
\mathcal{C}_p^\prime = A \mathcal{C}_p A^T
\end{align}

\mr{We can add here the explicit proof}

%
\subsection{KL decomposition}
%

Idea is to do a KL decomposition of prior and posterior covariances.

Write down the details.

This is covariant under parameter changes. Show it explicitly.

Is this the only decomposition that we can make?



%
\begin{acknowledgments}
%
We thank
XXX
for helpful discussions.
%
MR is supported in part by NASA ATP Grant No. NNH17ZDA001N, and by funds provided by the Center for Particle Cosmology.
%
Computing resources were provided by the University of Chicago Research Computing Center through the Kavli Institute for Cosmological Physics at the University of Chicago.
%
\end{acknowledgments}
%
\bibliographystyle{apsrev4-1}
\bibliography{biblio}
%
\end{document}
%
